\documentclass[12pt]{article}

\usepackage[utf8]{inputenc}
\usepackage{array}
\usepackage{fixmath}
\usepackage{booktabs}

\title{j}
\author{a.ibrahim }
\date{December 2019}

\renewcommand{\topfraction}{1.0}
\renewcommand{\bottomfraction}{1.0}
\renewcommand{\textfraction}{0.0}
\setlength {\textwidth}{7in}
\hoffset=-1.3in
\oddsidemargin=1.00in
\marginparsep=0.0in
\marginparwidth=0.0in                                                                              
\setlength {\textheight}{9.0in}
\voffset=-1.00in
\topmargin=1.0in
\headheight=0.0in
\headsep=0.00in
\footskip=0.50in                                        
\setcounter{page}{1}
\def\pos{\medskip\quad}
\def\subpos{\smallskip \qquad}
\newfont{\nice}{cmr12 scaled 1250}
\newfont{\name}{cmr12 scaled 1080}
\newfont{\swell}{cmbx12 scaled 800}

\begin{document}
\thispagestyle{empty}
\begin{center}
\textbf{PHYS  20323/60323: Fall 2019 - LaTeX Example}
%{\large \bf PHYSICS  20323/60323: Fall 2019 - LaTeX Example}
\end{center}

\noindent 
\begin{enumerate}
\item Consider a particle confined in a two-dimensional infinite square well  
\begin{center}
$V(x,y)=\Big\{$
\begin{tabular}{@{}c @{}l c}
    0 & , & $0\leq x\leq a, \textnormal{ } 0<y<a$\\
    $\infty$ & , & \textit{otherwise}
\end{tabular}
\end{center}

The eigenfunctions have the form:
$$\mathbf{\Psi} (x,y)=\frac{2}{a} \textnormal{ } \sin \Big( \frac{n\pi x}{a} \Big) \textnormal{ } \sin \Big( \frac{m\pi y}{a} \Big)$$

with the corresponding energis being given by:
$$E_{nm}=\Big( n^2+m^2\Big) \textnormal{ } \frac{\pi^2 \hbar^2}{2ma^2}$$

\begin{enumerate}
    \item (5 points) What are the levels of degeneracy of the five lowest energy values?
    \item (5 points) Consider a perturbation given by:
\end{enumerate}

$$\hat{H}' =a^2V_0 \hspace{.01in} \delta \hspace{.01in} \Big( x-\frac{a}{2} \Big) \textnormal{ } \delta \hspace{.01in} \Big( y-\frac{a}{2} \Big)$$

\hspace {.35in} Calculate the first order correction to the ground state energy.\vspace{.3in}

\item \textbf{The following questions refer to stars in the Table below.}\\
Note:  There may be multiple answers\\~\\
\begin{tabular}{|l|c|c|c|c|c|}
\hline
Name & Mass & Luminosity & Lifetime & Temperature & Radius\\
\hline
Zeta & 60. $M_{sun}$ & $10^6 L_{sun}$ & $8.0\times 10^5$ years & & \\
\hline
Epsilon & 6.0 $M_{sun}$ & $10^3 L_{sun}$ & & 20,000K & \\
\hline
Delta & 2.0 $M_{sun}$ & & $5.0\times 10^8$ years & & 2 $R_{sun}$ \\
\hline
Beta & 1.3 $M_{sun}$ & 3.5 $L_{sun}$ & & & \\
\hline
Alpha & 1.0 $M_{sun}$ & & & & 1 $R_{sun}$ \\
\hline
Gamma & 0.7 $M_{sun}$ & & $4.5\times 10^{10}$ years & 5000K & \\
\hline
\end{tabular}

\begin{enumerate}
    \item (4 points) Which of these stars will produce a planetary nebula at the end of their life.
    \vspace{.4in}
    \item (4 points) Elements heavier than \textit{Carbon} will be produced in which stars.
\end{enumerate}

\end{enumerate}
\end{document}